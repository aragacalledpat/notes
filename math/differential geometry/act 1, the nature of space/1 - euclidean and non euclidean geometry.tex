\documentclass{article}
\usepackage[margin=.5in]{geometry}
\usepackage[utf8]{inputenc}
\usepackage{amsthm}
\usepackage{setspace}
\setstretch{.75}
\title{1 - euclidean and non euclidean geometry}
\newtheorem{mydef}{Definition}


\newenvironment{andrew_section}[1]
    {
    \section{#1}
    \begin{itemize}
    }
    {
    \end{itemize}
    }

\begin{document}
\maketitle

\begin{andrew_section}{euclidean and hyperbolic geometry}
    \item flatness is defined as a space where the pythagorean theorem holds true.
    \item 
        euclidean geometry, parallel axiom: Through any point p not 
        on the line L, there exists exactly one line P that is parallel
        to L
    \item 
        1830, Lobachevsky announced the existence of hyperbolic geometry 
        by denying the parallel axiom.
    \item 
        hyperbolic geometry, hyperbolic axiom: There exists at least 2
        parallel lines through p that do not meet L
\end{andrew_section}

\begin{andrew_section}{spherical geometry}
    \item 
        parallel axiom states exactly one parallel exists.  denying 
        this means you're either stating there's less than 1 (aka 0) or
        more than 1.  the first option defines spherical geometry and the
        second defines hyperbolic
    \item 
        in spherical geometry, lines are "great circles", and these
        "lines" can be used to construct "triangles"
\end{andrew_section}

\begin{andrew_section}{the angular excess of a spherical triangle}
    \item 
        denying the parallel axiom is logically equal to denying
        that the interior angles of a triangle sum to pi radians
    \item  angular excess, $\mathcal{E}$ $\equiv (\textnormal{angle sum of triagle}) - \pi$ 
    \item 
        $\mathcal{E} = \frac{1}{R^2} \mathcal{A}$, for any triangle
        on a sphere with radius $\mathcal{R}$ and area $\mathcal{A}$
    
\end{andrew_section}

\begin{andrew_section}{intrinsic and extrinsic geometry of curved surfaces}
    \item 
        these geodesics can be thought of as length minimizing,
        but only if the distances between the two points are sufficiently
        small
    \item 
        this length minimizing property can then be used to define
        distance on a curved space.
    \item 
        now that we have defined distance we can define a geodesic
        "circle" on the curved surface by specifying a point and radius
    \item 
        intrinsic geometry: geometry that is knowable to a tiny ant
        living on the surface.  thus nothing will seem different to the
        ant, provided that distances all remain the same
    \item 
        extrinsic geometry: how the surface sits in space.  here you can
        alter things so distances are the same, but the global differences
        are still noted.
\end{andrew_section}

\begin{andrew_section}{constructing geodesics by their straightness}
    \item 
        to construct a geodesic on a surface, emanating from a
        point in direction v, stick out one end of a length of 
        narrow sticky tape down at p and unroll it onto the surface, 
        starting at direction v
    
\end{andrew_section}

\begin{andrew_section}{the nature of space}
    \item hyperbolic triangle is one where $\mathcal{E} < 0$
    \item 
        $\mathcal{E}(\Delta) = \mathcal{K}\mathcal{A} (\Delta)$,
        where $\mathcal{K}$ a constant that's positive for spherical
        geometry and negative for hyperbolic geometry
\end{andrew_section}


\end{document}