\documentclass[12pt, letterpaper, twoside]{article}
\usepackage[utf8]{inputenc}
\usepackage{amsthm}
\usepackage{setspace}
\setstretch{.75}
\title{1 - euclidean and non euclidean geometry}
\newtheorem{mydef}{Definition}


\newenvironment{andrew_section}[1]
    {
    \section{#1}
    \begin{itemize}
    }
    {
    \end{itemize}
    }

\begin{document}
\maketitle

\begin{proof}
    example proof for later
\end{proof}

\begin{mydef}
    example definition for later
\end{mydef}

\begin{andrew_section}{euclidean and hyperbolic geometry}
    \item flatness is defined as a space where the pythagorean theorem holds true.
    \item 
        euclidean geometry, parallel axiom: Through any point p not 
        on the line L, there exists exactly one line P that is parallel
        to L
    \item 
        1830, Lobachevsky announced the existence of hyperbolic geometry 
        by denying the parallel axiom.
    \item 
        hyperbolic geometry, hyperbolic axiom: There exists at least 2
        parallel lines through p that do not meet L
\end{andrew_section}

\begin{andrew_section}{spherical geometry}
    \item 
        parallel axiom states exactly one parallel exists.  denying 
        this means you're either stating there's less than 1 (aka 0) or
        more than 1.  the first option defines spherical geometry and the
        second defines hyperbolic
    \item 
        in spherical geometry, lines are "great circles", and these
        "lines" can be used to construct "triangles"
\end{andrew_section}

\begin{andrew_section}{the angular excess of a spherical triangle}
    \item 
        denying the parallel axiom is logically equal to denying
        that the interior angles of a triangle sum to pi radians
    \item  angular excess, $\mathcal{E}$ $\equiv (\textnormal{angle sum of triagle}) - \pi$ 
    \item 
        $\mathcal{E} = \frac{1}{R^2} \mathcal{A}$, for any triangle
        on a sphere with radius $\mathcal{R}$ and area $\mathcal{A}$
    
\end{andrew_section}

\begin{andrew_section}{intrinsic and extrinsic geometry of curved surfaces}
    \item test
\end{andrew_section}

\begin{andrew_section}{constructing geodesics by their straightness}
    \item test
\end{andrew_section}

\begin{andrew_section}{the nature of space}
    \item test
\end{andrew_section}


\end{document}