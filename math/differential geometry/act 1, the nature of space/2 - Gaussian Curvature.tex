\documentclass{article}
\usepackage[margin=.5in]{geometry}
\usepackage[utf8]{inputenc}
\usepackage{amsthm}
\usepackage{setspace}
\usepackage{esint}
\setstretch{.75}
\title{2 - Gaussian Curvature}


\newenvironment{andrew_section}[1]
    {
    \section{#1}
    \begin{itemize}
    }
    {
    \end{itemize}
    }

\begin{document}
\maketitle

\begin{andrew_section}{introduction}
    \item 
        $\mathcal{E} = + \frac{1}{\mathcal{R}^2}$ is called 
        \emph{Gaussian Curvature} of the sphere
    \item 
        $\mathcal{E} = - \frac{1}{\mathcal{R}^2}$ is also called 
        \emph{Gaussian Curvature}, in hyperbolic geometry
    \item 
        Gauss wanted to measure the curvature at a particular point,
        so we can rearrange 
        $\mathcal{E}(\Delta) = \mathcal{K}\mathcal{A} (\Delta)$ as
        $\mathcal{K} = \frac{\mathcal{E}(\Delta)}{\mathcal{A} (\Delta)}$,
        which makes it clear that $\mathcal{E}$ is 
        \emph{angular excess per unit area}.  to define the curvature
        at a point, we define a limit as the area of a triange around
        the point shrinks to 0.
    \item 
        there is nothing special about the triangle here, any n-gon will
        work, using the formula: \\ 
        $\mathcal{E}(\textnormal{n-gon}) \equiv [\textnormal{angle sum}]
        - (n - 2)\pi$
\end{andrew_section}

\begin{andrew_section}{the circumference and area of a circle}
    \item 
        we have defined curvature with respect to tiny triangles,
        but it is an abstract thing that has an "iron grip" over the geometry
        of a space.  two examples:
    \item 
        tiny circles determine curvature as well**
    \item 
        area of circles determine curvature as well**
\end{andrew_section}

\begin{andrew_section}{the local gauss-bonnet theorem}
    \item 
        angular excess is additive.  
        $\mathcal{E}(\Delta) = \mathcal{E}(\Delta_1) + \mathcal{E}(\Delta_2)$.
        this allows for finer and finer subdivisions, and thus a limit to the curvature 
    \item 
        $\mathcal{E}(\Delta) = \iint_\Delta \mathcal{K} \,d\mathcal{A}$
        
\end{andrew_section}


\end{document}