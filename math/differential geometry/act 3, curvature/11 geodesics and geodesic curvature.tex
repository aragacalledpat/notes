\documentclass{article}
\usepackage[margin=.5in]{geometry}
\usepackage[utf8]{inputenc}
\usepackage{amsthm}
\usepackage{setspace}
\usepackage{esint}
\usepackage{amssymb}
\setstretch{.9}

\newenvironment{andrew_section}[1]
    {
    \section{#1}
    \begin{itemize}
    }
    {
    \end{itemize}
    }

\title{11 - geodesics and geodesic curvature}

\begin{document}

\maketitle

\begin{andrew_section}{geodesic curvature and normal curvature}
    \item 
        curvature on a general surface can be broken up into two
        parts, one that is part of the surface, and thus invisible to
        the inabitants, and another part which is perpendicular to
        the surface, which the inhabitants can see.
    \item geodesic curvature  = $\kappa_g$ = visibile to an inhabitant... 
            normal curvature = $\kappa_n$ = invisible to an inhabitant
\end{andrew_section}

\begin{andrew_section}{meusnier's theorem}
    \item 
        Gauss
\end{andrew_section}

\begin{andrew_section}{geodesics are straight}
    \item 
        Gauss
\end{andrew_section}

\begin{andrew_section}{intrinsic measurement of geodesic curvature}
    \item 
        Gauss
\end{andrew_section}

\begin{andrew_section}{a simple way to measure geodesic curvature}
    \item 
        Gauss
\end{andrew_section}

\begin{andrew_section}{new explanation of sticky tape construction of geodesics}
    \item 
        Gauss
\end{andrew_section}

\begin{andrew_section}{geodesics on surfaces of revolution}
    \item 
        Gauss
\end{andrew_section}

\end{document}