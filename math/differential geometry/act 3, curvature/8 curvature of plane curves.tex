\documentclass{article}
\usepackage[margin=.5in]{geometry}
\usepackage[utf8]{inputenc}
\usepackage{amsmath}
\usepackage{amsthm}
\usepackage{setspace}
\usepackage{esint}
\usepackage{amssymb}
\setstretch{.9}

\newenvironment{andrew_section}[1]
    {
    \section{#1}
    \begin{itemize}
    }
    {
    \end{itemize}
    }

\title{8 - curvature of plane curves}

\begin{document}

\maketitle

\begin{andrew_section}{introduction}
    \item 
        two dimentional surface can be deformed in some limited ways
        while preserving intrisinc geometry, one dimentional surfaces
        (curves) can be arbitrarily deformed while preserving the
        intrinsic geometry....so to an ant on a curve, the shape of the
        curve has no meaning
    \item 
        extrinsic curvature of 1-d curves will have direct meaning for
        intrinsic curvature of 2d surfaces
    \item 
        if we imagine a bead on the 1d curve, and imagine the bead
        follows phyiscal laws, when a turn happens, the bead feels 
        an acceleration because the curvature implies a change of
        velocity.
\end{andrew_section}

\begin{andrew_section}{the circle of curvature}
    \item 
        newton was the first to examine curvature, and did so
        when he was 21.  first he found a way to determine
        the 'tangent circle'. at a point to the curve.  much like
        tangent line at a point, but instead it is to describe the
        curvature at a point.  basically take two perpendicular lines
        near the point, note where they directionally meet, and take the
        limit as the two points come together...similar to tangent lines.
        where they meet is the center of the cicle.  the radius of the circle
        is k.
    \item 
        newton used that to find sigma, the amount the tangent lines
        diverage 
\end{andrew_section}

\begin{andrew_section}{newton's curvature formula}
    \item 
        if $y = f(x)$ is the curve, and the x-axis is parallel
        to the tangent at a point, the curvature is just the second
        derivative of the function (follows from taylors theorem).
    \item 
        newton found that if you place the x axis at an arbitary point
        instead, you just have to patch the formula.  looks interesting
\end{andrew_section}

\begin{andrew_section}{curvature as rate of turning}
    \item 
        after newton, in 1761, someone came up with a more flexible
        definition of curvature. rate of turning of the tangent
        with respect to arc length.  $\phi$ is angle of tangent change,
        so $\kappa = \frac{d\phi}{ds}$.
    \item 
        totally local definition, don't have to shuffle around normals
    \item 
        now look at things from the perspective of 2 unit normals,
        T and N. picture them wiggling in place, subtracting the 
        difference. these are unit vectors so when they wiggle, they
        wiggle on the unit circle.  i think here you are supposed to
        view it as just a trig thing, that the wiggle of one has a trig
        relationship to the wiggle of the other, and we also know the curvature
        of the circle so bring that in.
    \item 
    take v to be
    $\begin{pmatrix}
        \dot x \\
        \dot y
        \end{pmatrix}$
    and now we can do our calculations with just a parameterized curve
    instead of an explicity function, which is very useful.  e.g.
    $\tan{\phi} = \frac{\dot y}{\dot x}$
\end{andrew_section}

\begin{andrew_section}{example: newton's tractrix}
    \item 
        Gauss
\end{andrew_section}

\end{document}