\documentclass{article}
\usepackage[margin=.5in]{geometry}
\usepackage[utf8]{inputenc}
\usepackage{amsthm}
\usepackage{setspace}
\usepackage{esint}
\usepackage{amssymb}
\setstretch{.9}

\newenvironment{andrew_section}[1]
    {
    \section{#1}
    \begin{itemize}
    }
    {
    \end{itemize}
    }

\title{6 - isometries and complex numbers}

\begin{document}

\maketitle

\begin{andrew_section}{introduction}
    \item 
        isometries preserve magnitude of angles.  ones
        that preserve the sense of the angle are called
        \emph{direct isometries} and ones that reverse the
        sense are called \emph{opposite isometries}. direct are a
        special kind of conformal mapping and opposite are a special
        kind of anticonformal.
    \item 
        isometries of a surface form a group $\mathcal{G}(\mathcal{S})$
    \item 
        direct and opposite compose like + and - under multiplication,
        e.g. (-) (-) = +. thus, direct isometries are a subgroup of 
        $\mathcal{G}(\mathcal{S})$, denoted 
        $\mathcal{G}_+(\mathcal{S})$, but opposites don't (closure
        is not respected in the example above)
    \item 
        let's take 2 opposite isometries.  let one be fixed, denote 
        it by $\xi$, and let another be the general opposite
        isometry, varying over all possible ones on the surface...
        call  it $\zeta$.  the inverse of an opposite transformation
        is also opposite, so $\xi^{-1}$ is opposite.  via the 
        (-)(-) = (+) example, 
        $\xi^{-1} \circ \zeta \in \mathcal{G}_+(\mathcal{S})$.  thus 
        $\zeta \in \mathcal{G}_+(\mathcal{S}) \circ \xi$
    \item 
        its clear that by fixing one opposite, and left multiplying by g+,
        we get the full group of opposite isometries.  now we can construct
        the full group g by combining the direct isometries with the opposite
        isometries.
    \item 
        $\mathcal{G}(\mathcal{S}) = $ direct union opposite \\
        $\mathcal{G}(\mathcal{S}) = [\mathcal{G}_+(\mathcal{S})] \bigcup [\mathcal{G}_+(\mathcal{S}) \circ \xi]$ 
    \item 
        not every surface has meaningful isometries
    \item 
        **three geometries of constant curvature have symmetry
        groups $\mathcal{G}_+(\mathcal{S})$ that are subgroups
        of mobius transformations of complex plane, which are
        $\frac{az+b}{cz+d}$**

\end{andrew_section}

\begin{andrew_section}{mobius transformations}
    \item 
        can be decomposed into 4 simple transformations:\\
        translate, rotate/scale, inversion, another translation.
        this can make proving and understanding certain properties
        much more straightforward. transformation is called singular
        when ad-bc=0
    \item 
        $z \rightarrow \frac{1}{z}$, complex inversion.  key to
        understanding mobius transformations.  view in polar coordinates:\\
        $r e^{i\theta} \rightarrow (\frac{1}{r}) e^{- i \theta}$.
        invert the radius, negate the angle.  can be viewed as two
        step process: take reciprical lengh, complex conjuate. order actually
        doesn't matter.  inverting the length is called \emph{geometric inversion}.
        unit circle is very important to the geometric inversion.  sends
        points interior to exterior and vice a versa, leaves circle untouched.
        denote this transformation $\mathcal{J}_C$. we can generalize the
        transform to other circles of different center and radius, then
        denote it $\mathcal{J}_{K}$, or whatever.
    \item 
        *inversion is just a rotation of the riemann sphere by $\pi$,
        about the real axis* \\
        inversion is anticonformal, preserves circles \\
        if a circle inside C passes through 0, it gets mapped to a line
        and its interior becomes like a half plane, split by the line. \\
        this correspondence between half-planes and interior circles is 
        how half-plane model of hyperbolic geometry is mapped to the
        beltrami-pointcare disc(nice...).
    \item 
        we can now see that mobius transformations are conformal
        and map circles to circles, region to the left of the circle
        is mapped to the left of the circle after the transform
    \item 
        we've geometrically made it rigourous that $\frac{1}{\infty} = 0$
        and $\frac{1}{0} = \infty$, to make it algebraically clear,
        let a point of the riemann sphere be defined by 2 complex numbers,
        [like spinors...cool].  z = one/the other.  one point on the 
        riemann sphere has one z in C, but one z in C has infinite pairs on
        riemann sphere.  now infinity just equals [one of the pairs,0]
    \item 
        mobius transformation can be thought of as a 2x2 matrix,
        linear transformation with complex constants a,b,c,d.  this helps
        to find compositions of 2 mobius transformations (matrix multiplication),
        and inverse of a mobius transformation (inverse of matrix).
        these matrices are not unique, but we can normalize them.  then
        they are unqiue up the the sign....so M will always be the same as
        -M
\end{andrew_section}

\begin{andrew_section}{the main result}
    \item 
        Gauss
\end{andrew_section}

\begin{andrew_section}{einstein's spacetime geometry}
    \item 
        Gauss
\end{andrew_section}

\begin{andrew_section}{three dimentional hyperbolic geometry}
    \item 
        Gauss
\end{andrew_section}

\end{document}