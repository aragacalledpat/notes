\documentclass{article}
\usepackage[margin=.5in]{geometry}
\usepackage[utf8]{inputenc}
\usepackage{amsthm}
\usepackage{setspace}
\usepackage{esint}
\usepackage{amssymb}
\setstretch{.9}

\newenvironment{andrew_section}[1]
    {
    \section{#1}
    \begin{itemize}
    }
    {
    \end{itemize}
    }

\title{4 - Mapping Surfaces: The Metric}

\begin{document}

\maketitle

\begin{andrew_section}{introduction}
    \item 
        Gauss' big insight was to show that the intrinsic geometry of
        a surface is totally determined by having a rule for infinitesimal
        distances between two points; the metric.  this can thus also determine
        length's of curves, and geodesics which are paths that minimize the distance
    \item 
        in this context, map means 'cartographic' map, where 'mapping'
        will be the more typical mathematical usage.  
    \item 
        stragegy is to make a one to one function between: \\
         points $\hat{z}$ on $\mathcal{S}$, to points z on $\mathbb{C}$.  this will
         inevitably cause some distortion
    \item 
        take points $\hat{z}$, $\hat{q}$ on $\mathcal{S}$.  \\
        represent them as z, q on $\mathbb{C}$ with z = $r{\rm e}^{i\theta}$,
        q = z + $\delta z$ \\
        $\delta \hat{s}$ is distance between $\hat{z}$ and $\hat{q}$ \\
        $\delta s = |\delta z|$
    \item 
        rule giving  $|\delta z|$ is the metric.  obviously depends on both
        direction and length 
    \item 
        $d \hat{s} = \Lambda (z, \gamma) ds$ \\ 
        given a point and a direction, how much do we have to locally
        expand $\mathbb{C}$ to preserve distances?
\end{andrew_section}

\begin{andrew_section}{projective map of the sphere}
    \item 
        imagine the southern hemisphere of the sphere, a bowl shape.
        the south pole, the bottom of the bowl, rest on the origin of
        $\mathbb{C}$. the center of the sphere, shoots out light rays.
        where those light rays hit $\mathbb{C}$ is called the 
        \emph{projective map of the southern hemisphere}
    \item 
        this map sends circles on $\mathcal{S}$ to ellipses on $\mathbb{C}$.
        this is true in general for any surface, if the radius of the circle is
        infinitesimal.
    \item 
        projective map sends geodesics on $\mathcal{S}$ to lines on 
        $\mathbb{C}$, but it does not preserve angles.
    \item 
        formula for metric on the sphere, given polor coordiantes on
        $\mathbb{C}$

\end{andrew_section}

\begin{andrew_section}{the metric of a general surface}
    \item 
        different maps have different different metrics even though
        they describe the same intrinsic geometry.  for example,
        imagine a map of the sphere where longitude and latitude 
        ($\theta,\phi$), we map to the carteasan coordiantes
        ($\theta,\phi$).  the metric for this map will be
        $ d \hat{s}^2  = R^2[\sin^2 (\phi) d\theta^2 + d \phi^2]$,
        which is different from the projective map metric for a sphere.
    \item 
        take a general surface and draw 2 families of curves, that both
        vary smoothly, and so any point on the surface can be uniquely represented
        by a point on each of the curves (let's call them U curves and V curves).
        let a point on the surface be labeled as $u + i v$ and a point can
        be labeled as $\hat{z} = U + i V$
    \item 
        imagine a small movement away from z on the map.  
        $\delta z = \delta u + i \delta v$ \\
        but this is on the map, how do we project it back to the actual
        surface? 
    \item by virtue of differentiability, we can say that some small movement
        on the v curve will produce some small movement on the surface.
        put another way, $\frac{\partial \hat{s_1}}{\partial u} \equiv A$,
        $\frac{\partial \hat{s_2}}{\partial v} \equiv B$
    \item 
        we can see that A and B are the local scale factors that have to be
        applied to the map for distances to be preserved.  A can be viewed
        as inversly proportional to the crowding of the u-curves.  the greater
        the crowding, the greater result $\delta \hat{s_1}$ will have on u
    \item 
        $\omega$ is angle between u-curves and v-curves, which depends on
        position
    \item 
        general metric for the surface: \\
        $d \hat{s}^2 = A^2 du^2 + B^2 dv^2 + 2Fdu \ dv$
    \item 
        however, once the u-curves are chosen, it is always possible
        to select an orthoganal set of v-curves, which anniliates the
        last term.
    \item 
        in general it is impossible to cover all of the surface with
        a single u-v set of curves - the curves will inevitably intersect on
        any closed surface 
\end{andrew_section}

\begin{andrew_section}{the metric curvature formula}
    \item 
        once handed a metric, you should in principle be able to derive the
        curvature at any point.  what is not a given is that the formula is
        beautiful and simple.  it will take most of the book to derive for real.
    \item 
        *** $\mathcal{K} = - \frac{1}{AB}(\partial_v [\frac{\partial_v A}{B}] + \partial_u [\frac{\partial_u B}{A}])$
        
    \item 
        can also be used to calculate areas. $ dA = ABdu \ dv$ 
\end{andrew_section}

\begin{andrew_section}{conformal maps}
    \item 
        projective map preserves straight lines but usually it's better 
        to preserve angles.
    \item 
        map that preserves angles and sense is \emph{conformal},
        map that preserves angles and inverts sense ins \emph{anti-conformal}
    \item 
        angle between curves means the angle between their tangents
    \item 
        with respect to metric formula, a map is conformal if the scale
        factor, $\Lambda$, depends on position but not direction.
        this way, *infinitesimal shapes on the map are the same shape, just
        a different size!*
    \item 
        gauss proved that given any surface, you can find an orthogonal map
        that is also conformal, in the sense that A = B.  since A is the same as B
        and the coordinates are orthogonal, we can express the general metric
        of the surface as \\
         $ d\hat{s}^2 = \Lambda^2 [du^2 + dv^2]$.
        \item 
            this also beatifully simplifies the curvature formaula to: \\
            $\mathcal{K} = - \frac{\nabla^2 \ln \Lambda}{\Lambda^2}$
\end{andrew_section}

\begin{andrew_section}{some visual complex analysis}
    \item 
        every surface contains infinite variety of u-v curves and conformal maps 
    \item 
        let a conformal mapping be $F = \mathbb{C} \rightarrow \mathcal{S}$
    \item 
        conformal $(\tilde{u}, \tilde{v})$-coordinates can be created by
        rotating/expanding/translating any given u-v curves.
    \item 
        let $z = u + iv$ be in $\mathbb{C}$ and 
        $\tilde{z} = \tilde{u} + i \tilde{v}$ be a seperate copy of $\mathbb{C}$
        but under some function f.
    \item 
        big idea is that before we had f going from S to C, now we also have a
        function F going from C to S.  so all together we have a copy of C
        mapping to S, which then maps onto another C.  C -> S -> C.
        both are conformal.  we now have a freedom of rotating/expanding/translating
        the u-v lines in the first copy of C, which will give us new u-v coordinates
        in S, thus leading to an infinite variety of them.
    \item 
        $f(z) = a e ^ {i \tau} z + w$. scale by a, twist by tau, shift by a complex
        contstant w.
    \item 
        remember we have F as the map and now f as a function on the first copy
        of C.  so our new u-v coordinates on S are given by $\tilde{F} = F \circ f$
    \item  
        useful to think of the derivative which blows away the constant w, and is just the 
        amplification + twist * (tiny move in z).  "amplitwist"
    \item 
        from some basic reasoning (?) we can see that differentiable complex mappings
        are all conformal.
    \item 
        this procedure allows one to take any conformal mapping from C to itself
        and "pass it through" S.
\end{andrew_section}

\begin{andrew_section}{the conformal stereographic map of the sphere}
    \item 
        f
\end{andrew_section}

\begin{andrew_section}{stereographic formulas}
    \item 
        f
\end{andrew_section}

\begin{andrew_section}{stereographic preservation of circles}
    \item 
        f
\end{andrew_section}

\end{document}