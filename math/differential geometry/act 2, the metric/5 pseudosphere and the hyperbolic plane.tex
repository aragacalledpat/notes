\documentclass{article}
\usepackage[margin=.5in]{geometry}
\usepackage[utf8]{inputenc}
\usepackage{amsthm}
\usepackage{setspace}
\usepackage{esint}
\usepackage{amssymb}
\setstretch{.9}

\newenvironment{andrew_section}[1]
    {
    \section{#1}
    \begin{itemize}
    }
    {
    \end{itemize}
    }

\title{5 - Pseudosphere and the hyperbolic plane}

\begin{document}

\maketitle

\begin{andrew_section}{Belatrami's Insight}
    \item 
        two areas of discovery were reaching a head at the same time.
        hyperbolic geometry by lobachevsky, and gauss' differential geometry.
        both around 1830.
    \item 
        1868, belrimi thought there might be a connection. hyperbolic
        geometry was 40 years old and obscure.  [in some way, this makes
        sense to me- he was legit just denying one of the euclidean axioms
        and showing it produced cool stuff].
    \item 
        beltrimi reasoned via the local gauss-bonnet theorom
        that if he found a surface of constant negative curvature
        it would obey the laws of hyperbolic geometry.
\end{andrew_section}

\begin{andrew_section}{The Tractrix and the pseudosphere}
    \item 
        psuedosphere has this constant negative curvature, but it has
        a boundary...which is something of a problem because the hyperbolic
        plane was conceived to be infinite
    \item 
        \emph{tratrix}, investigated by newton.  imagine a table (axes X and Y),
        with a paper weight on it, and a string of length R attatched to it.
        the other end of the string touches the end of the table.  drag the
        string around the \emph{edge} of the table. curve the weight makes
        is a tractrix.  the weight will hypothetically never reach the 'Y-axis'.
    \item 
        let $\sigma$ represent the arc length, so $\sigma=0$ would be
        before you start pulling the string at all.  [nifty triangle similarty
        argument between delta sigma/-delta X and X/R...second triangle being
        the big one made by the string and distance from Y axis, first one
        obviously being infinitesimal]: \\ $X = Re^{-\frac{\sigma}{R}}$
    \item 
        pseudosphere is the surface of revolution for a tratrix
\end{andrew_section}

\begin{andrew_section}{A Conformal Map of the Pseudosphere}
    \item 
        need to create a conformal map from pseudosphere to $\mathbb{C}$.
        let x be the angle and sigma the arc length of the tractrix.
        so $(x,\sigma)$ will be sent to $x + yi$
    \item 
        a few considerations lock you into the bounded circle approach.
    \item 
        generators are orthogonal so the mapping must be orthogonal.
    \item 
        take a circle around the pseduosphere of sigma = constant.  this
        circle will have a radius X.  now take a triangle of two lines Xdx,
        $d\sigma$, because of the metric, this has to be shrunk to a trianle of
        lines dx dsigma.
    \item 
        the map is $d\hat{s} = \frac{R ds}{y}$, where $ds = \sqrt{dx^2 + dy^2}$
\end{andrew_section}

\begin{andrew_section}{the beltrami-poincare half-plane}
    \item 
        it is typical to set the radius of the pseudosphere
        to 1 so the curvature is -1.
    \item 
        it was important to beltrami that the pseduosphere be generalized
        to be infinite, like real hyperbolic geometry.  imagine a paint roller
        of radius 1.  after one revolution on a wall you have painted a strip
        $2\pi$ thick.  now we use our pseudosphere as a roller on the plane.
        horizontal strips represent lines of constant$\sigma$, going around
        the one circular strip of the pseduosphere.
    \item 
        that removes the lack of inifinitude, but it still has a rim that
        must be removed.  one can easily use the conformal mapping and the known
        metric to extend the rim further - you know exactly the scale factor
        thanks to the metric so you just keep going.
    \item 
        standardized hyperbolic metric: $d\hat{s} = \frac{ds}{y}$
    \item 
        this really makes for a half-plane.  the tractrix's 'end' at 
        y = 0.  if you imagine a particle traveling in this space,
        it will slow down more and more the closer you get to
        y = 0.
    \item 
        thus we now have \emph{The Hyperbolic Plane} $\mathbb{H}^2$,
        defined on y > 0, with the metric above.
    \item 
        to imagine how the metric affects things like arc length,
        we can imagine painting the half plane with circles of radius
        $\epsilon$, and letting them scale.  the length of the path
        is how many balls it goes through - way more balls towards the origin
    \item   
        this all makes sense - but how can we explain how geodesics
        are semicircles that hit the original at a right angle?
\end{andrew_section}

\begin{andrew_section}{using optics to find geodesics}
    \item 
        Gauss
\end{andrew_section}

\begin{andrew_section}{the angle of parallelism}
    \item 
        Gauss
\end{andrew_section}

\begin{andrew_section}{the beltrami-poincare disc}
    \item 
        Gauss
\end{andrew_section}

\end{document}